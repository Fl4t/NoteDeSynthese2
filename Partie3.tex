\chapter{Mise en \oe{}uvre d'IITAMP} % (fold)
\label{cha:Mise en oeuvre d'IITAMP}

\begin{it}
Après la création d'IITAMP ainsi que l'élaboration de toute sa
documentation, je l'ai installé sur un serveur local chez un client pour
héberger la solution GeMa qui est développé par \emph{IdentIt}. J'ai eu
aussi l'opportunité de réaliser une vidéo d'utilisation de GeMa dans le
cadre d'une présentation par les responsables de l'entreprise cliente à
leurs salariés.
\end{it}

\section{Présentation de GeMa} % (fold)%{{{
\label{sec:Présentation de GeMa}

\lettrine{G}{eMa} est une solution de maintenance innovante alliant
RFID, mobilité et technologie Web. L'application embarquée sur
PDA,\footnote{\emph{Personal Digital Assistant} où assistant numérique
personnel.} permet la consultation des interventions à réaliser, la
saisie des rapports et l'accès aux données techniques nécessaires au bon
déroulement des travaux.

\begin{quotation}
\og{}Avec GeMa, le responsable de maintenance prépare depuis son poste
informatique les ordres de travaux pour lesquels il mandate un
intervenant sur site. Les techniciens, quant à eux, reçoivent en temps
réel leurs missions sur PDA qu'ils complètent grâce à une interface
ergonomique (menus déroulants, cases à cocher\dots). Les rapports ainsi
réalisés remontent instantanément sur le serveur. \fg{} M.~Dubourg.
\end{quotation}

Une intervention à réaliser sur un équipement : le technicien
l'identifie grâce à son étiquette RFID scannée par le PDA. Une anomalie
découverte lors d'un contrôle : le technicien la photographie grâce à
l'appareil numérique intégré. Un élément laissé sur place : le
technicien indique sa position sur site, grâce aux fonctions GPS de
GeMa. En temps réel, les responsables et/ou clients disposent
d'informations fiables et exploitables au travers d'historiques, de
statistiques, de bilans, d'alertes email ou via des échanges avec un
logiciel tiers.

\begin{figure}
  \begin{center}
    \includegraphics{images/gema.png}
    \caption{Schéma simpliste d'utilisation de GeMa.}
    \label{workflow}
  \end{center}
\end{figure}
% section Présentation de GeMa (end)%}}}

\section{Présentation du client Syngenta} % (fold)%{{{
\label{sec:Présentation du client Syngenta}

Le groupe Syngenta est un des leaders mondiaux de l’agrofourniture.
Ce groupe développe une approche par culture reposant sur 2 piliers,
d'une part la création, le développement et la commercialisation de
variétés pour les productions agricoles majeures, d'autre part la
protection des plantes.

\begin{description}

  \item[production de variétés pour l'agriculture :] betterave à sucre,
    céréales à paille (blé tendre, blé dur, orge de printemps, orge
    hybride), fleurs, légumes (pour le marché du frais et de
    l’industrie), maïs, oléagineux (tournesol et colza) et les cultures
    légumières.

  \item[Le traitement des plantes :] Une combinaison de solutions
    phytopharmaceutiques contre les principaux ravageurs et nuisibles
    des cultures à base de produits de protection des semences,
    d’herbicides, de fongicides et d’insecticides mais aussi d’insectes
    auxiliaires.

\end{description}
% section Présentation du client Syngenta (end)%}}}

\section{Mise en place de la solution} % (fold)%{{{
\label{sec:Mise en place de la solution}

Pour une interconnexion entre les sites, un dispositif de sécurité
impressionnant à été mis en place par Syngenta. En effet, ceux-ci
utilise la technologie VPN,\footnote{\emph{Virtual Private Network} ou
réseau privé virtuel.} qui permet d'obtenir une liaison sécurisée à
moindre coût.

Le réseau privé virtuel vise à apporter certains éléments essentiels
dans la transmission de données : l'authentification (et donc
l'identification) des interlocuteurs, la confidentialité des données via
le chiffrement (qui vise à les rendre inutilisables par quelqu'un
d'autre que le destinataire).

La mise en \oe{}uvre d'IITAMP ainsi que de GeMa s'effectuant dans
l'usine de Nerac près de bordeaux, le client nous à prêter un ordinateur
portable avec tout les outils nécessaire à la connexion à distance pour
éviter de me déplacer de Dunkerque jusqu'au site, ce qui représente à
peu près 600 kilomètres...

\begin{figure}
  \begin{center}
    \includegraphics[scale=0.5]{images/vpn.png}
    \caption{Exemple de réseau privé virtuel entre deux sites.}
    \label{vpn}
  \end{center}
\end{figure}

\subsection{Harmonisation des systèmes d'informations} % (fold)%{{{
\label{sub:Harmonisation des systèmes d'informations}

Dans le cadre de cette mise en \oe{}uvre, le client veut récupérer les
données des interventions qui sont stockés dans la base MySQL de GeMa
dans leur système d'information qui possède une base de donnée SQL
Server. Cette duplication de donnée permet une intégration des
informations récoltés par GeMa dans le SI de l'entreprise à des fins de
traçabilité, ce qui permet à partir des autres logiciels du client de
faire des statistiques de production.

Pour cette duplication, il à fallu que je développe un script PHP qui
tout d'abord interroge la base de donnée de GeMa pour récupérer les
données, qui ensuite traite ses données pour les rendre compatibles à la
structure de la base du client ainsi qu'a la technologie SQL Server
utilisée et qui enfin insère ses données traités dans la base du client.

\subsection{Les tests} % (fold)%{{{
\label{sub:Les tests}

Comme nous l'avons vu, GeMa utilise la mobilité pour le rapport des
interventions. Pour tester si l'application GeMa fonctionne correctement
ainsi que le script de duplication de donnée, quelques tests furent
nécessaire. Étant donnée que le site de production n'est pas proche,
l'utilisation d'un émulateur de PDA au travers du réseau VPN fut
nécessaire.\\

Quelques problèmes sont apparu pendant la phase de debogage :
\begin{description}

  \item[Communication entre les bases :] XAMPP par défaut ne permet pas
    d'utiliser une base SQL Server, il à fallut que j'installe et
    configure le driver officiel de l'entreprise
    Microsoft~\textregistered pour permettre l'accès à la base de
    données.

  \item[Synthaxe invalide pour SQL Server :] Une habitude est ancrée
    dans l'utilisation du logiciel libre MySQL qui est de forcer
    l'encodage des caractères en UTF-8 afin d'éviter de futur problèmes.
    Cependant SQL Server ne connait pas la syntaxe SQL permettant cela.
    GeMa étant pourvu d'un système d'alerte par mail, M.~Dubourg
    recevait un courriel à 5 minutes d'intervalles lorsque l'émulateur
    d'assistant personnel lançait la synchronisation distante avec le
    serveur,\footnote{fonction de synchronisation automatique de GeMa
    Mobile lorsque il est sur son socle}, sont client mail à bien
    entendu déplacer l'expéditeur entant que spam au bout de plusieurs
    reprises. Ne fermant pas l'émulateur la nuit pour éviter d'avoir à
    le relancer, une quantité impressionnante de mail m'a été annoncé
    quelques jours plus tard lorsque le cogérant à inspecté sa boite à
    pourriels.

  \item[Obligation de renseigner tout les champs :] L'ancien système de
    relevés ce faisant à l'aide d'un papier/crayon puis par l'ajout de
    ses annotations dans un fichier Excel, lorsqu'un champ n'avait
    aucune valeur, un zéro lui était attribué. En inspectant les données
    de la base SQL Server, je me suis rendu compte que mon script quant
    à lui n'insérait rien lorsqu'il n'y avait pas de valeur. J'en ai
    déduit qu'aucune règle de gestion n'avait été crée afin d'empêcher
    la non-saisi. J'ai donc été obligé d'inscrire cette règle via le
    code,\footnote{c'est la base du client, je n'ai pas à intervenir
    dessus même si elle me parait mal conçu.} alors que la bonne méthode
    est habituellement de le faire via une option lors de la création de
    la base.

\end{description}
% subsection Les tests (end)%}}}
% subsection Harmonisation des systèmes d'informations (end)%}}}
% chapter Mise en oeuvre d'IITAMP (end)
