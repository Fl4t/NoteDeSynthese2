\chapter{Les entreprises} % (fold) \label{cha:Les entreprises}

\begin{it}

Ce chapitre illustre les entreprises dans lequels j'ai évolué. Dans un
premier temps \emph{IdentIt} m'a acceuilli pour le premier stage puis
dans un second temp ce sont les deux entreprises \emph{IdentIt-A2SI}. Je
présente aussi l'entreprise \emph{Syngenta} qui acceuil mon projet de
stage IITAMP qui est décrit dans le chapitre \ref{cha:Mise en oeuvre
d'IITAMP}.

\end{it}

\section{Les entreprises d'accueils} % (fold)
\label{sec:Les entreprises d'aceuils}

\subsection{IdentIt} % (fold)
\label{sub:IdentIt}

\emph{IdentIt} est une SARL d'édition de logiciels spécialisé dans
l'informatique industrielle et des systèmes d'identification par
radio-fréquence RFID\,\footnote{\emph{Radio Frequency IDentification.}}.
Elle est composée de quatre personnes et se situai à Petite-Synthe
durant mon premier stage. \bsc{M.~Dubourg} et \bsc{M.~Lesage} sont les
fondateurs de cette structure, le premier s'occupe de la partie
technique en tant que chef de projet et le second de la partie gestion
de l'entreprise. Ils sont accompagnés par deux développeurs, l'un
travaillant sur des solutions Windows Mobile, l'autre étant plus
spécialisé sur le développement Web.

% subsection IdentIt (end)

\subsection{A2SI} % (fold)
\label{sub:A2SI}

\emph{A2SI}, implantée au coeur du tissu industriel dunkerquois, est une
entreprise spécialisée dans l’étude et la réalisation de projets
d’automatisation, de régulation et d’instrumentation de process
industriels. Son équipe est composée de 3 ingénieurs et de 3
techniciens, met ses atouts au service de la qualité du travail et de la
satisfaction des clients, tant en France qu’à l’étranger. Cette
entreprise ce situai dans la rue de la république à Saint Pol Sur Mer.

% subsection A2SI (end)

\subsection{La fusion} % (fold)
\label{sub:La fusion}

\emph{IdentIt} est une sous-cursale d'\emph{A2SI} qui fut crée pour
répondre aux besoins de solutions informatiques toujours plus grand dans
les industries. Ceux-ci ont tout d'abord enménagé à Petite-Synthe
pendant deux ans dans une pièce d'approximativement vingt-cinq mètres carrées
pour quatre personnes ce qui n'était pas évident lorsque l'on acceuil
une personne supplémentaire, nous étions un peu sérré.

Au cours du deuxième stage, les entreprises ont \og fusionné \fg{} dans
de nouveau locaux dans la zone industrielle de Saint Pol Sur Mer. Cela
ma permis de vivre une fusion d'entreprise, d'évoluer dans un grand
batiment neuf d'a peu près cinq-cent mètres carrées ainsi que de
connaitres de nouvelles personnes.

% subsection La fusion (end)

% section Les entreprises d'aceuils (end)

\section{L'entreprise cliente} % (fold)
\label{sec:L'entreprise cliente}

Le groupe Syngenta est un des leaders mondiaux de l’agrofourniture.
Ce groupe développe une approche par culture reposant sur 2 piliers,
d'une part la création, le développement et la commercialisation de
variétés pour les productions agricoles majeures, d'autre part la
protection des plantes.

\begin{description}

  \item[production de variétés pour l'agriculture :] betterave à sucre,
    céréales à paille (blé tendre, blé dur, orge de printemps, orge
    hybride), fleurs, légumes (pour le marché du frais et de
    l’industrie), maïs, oléagineux (tournesol et colza) et les cultures
    légumières.

  \item[Le traitement des plantes :] Une combinaison de solutions
    phytopharmaceutiques contre les principaux ravageurs et nuisibles
    des cultures à base de produits de protection des semences,
    d’herbicides, de fongicides et d’insecticides mais aussi d’insectes
    auxiliaires.

\end{description}

% section L'entreprise cliente (end)

% chapter Les entreprises (end)
