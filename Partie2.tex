\chapter{Création du serveur IITAMP} % (fold)
\label{cha:Création du serveur IITAMP}

Je suis tout d'abord parti d'une base WAMPP générique. C'est un logiciel
libre développer par une équipe de bénévoles. Ce logiciel permet de
mettre en \oe{}uvre un serveur web sur un PC Windows. La suite WAMPP
regroupe les logiciels Apache(Serveur HTTP), MySQL(serveur de base de
données) et PHP(langage de programmation). De ce fait, il permet
d'héberger sur la machine qui la compose un site internet dynamique. La
solution de base ne nous satisfesant pas, mon travail consista a
modifier ce gratuiciel.

La première étape consista a définir une arborescence de dossiers qui
nous permetterait une flexibilité dans plusieurs domaines :
- la maintenance de bug
- l'évolutivité
- la pérénité des données
- l'éfficacité
- la simplicité

Pour proposer une maintenance de bug éfficace est facile il a fallut
installer le module php xdebug qui permet d'afficher des érreurs plus
sinificative que celle de php par defaut. En effet, celui-ci permet la
coloration synthaxique ainsi que la presentation du déroulement
précédent le bug. J'ai aussi externaliser les fichiers de log dans un
dossiers séparé ce qui permet d'y acceder facilement. Auquel cas on
puisse demander au client de nous envoyer le dossier archivé par mail
pour analyser si xdebug ne suffirait pas.

comme les dossiers sont séparés méticuleusement, une mise à jour de
l'applicatif web ou du serveur web ce fait à partir en quelques étapes
très simple contrairement au logiciel de base qui aurait nécéssité
beaucoup de manipulation.

Les bases de données ainsi que les informations du navigateur web sont
externalisé du dossier IITAMP ce qui permet de faire une sauvegarde
éfficace de celles-ci. La sauvegarde ce fait à partir d'une tache
journalière que j'ai décrite dans un document pour l'utilisateur.

Pour optimiser le serveur IITAMP, une inspection profonde du logiciel
fut nécessaire. En effet, celui-ci est composé de base d'une mutitude de
librairies et modules qui ne sont pas forcement utile au applications
développer par l'entreprise. De ce fait, un néttoyage fut entrepris. Un
gain de 20 mega-octets au final.

Pour rendre simple l'utilisation du serveur web pour le client, il à
fallut créer des scipt d'automatisation de certaines taches pas
forcement évidente pour les non initiés. Par exemple l'installation des
applications entant que services. Le démarrage des dit services. Aussi,
le developpement d'une page d'index dans le cas ou le client à plusieurs
applications web de l'entreprise IdentIt.
% chapter Création du serveur IITAMP (end)
