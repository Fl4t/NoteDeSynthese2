\chapter{Introduction} % (fold)
\label{cha:Introduction}

\lettrine{J}{e} m'appelle Julien \bsc{Stechele}, je suis actuellement en
BTS\, \footnote{\emph{Brevet de Technicien Supérieur.}} informatique de
gestion option développeur d'applications et dans le cadre de mes études
j'ai eu la chance d'effectuer deux stages durant cette formation. Mon
stage de première année s'est déroulé du 16 mai 2011 au 8 juillet 2011
dans la société \emph{IdentIt} ainsi que mon deuxième stage qui lui
s'est déroulé du 3 janvier 2012 au 21 février 2012.

C'est une SARL d'édition de logiciels spécialisé dans l'informatique
industrielle et des systèmes d'identification par radio-fréquence
RFID\,\footnote{\emph{Radio Frequency IDentification.}}. Elle est
composée de quatre personnes et se situe depuis peu à Saint Pol sur Mer.
\bsc{M.~Dubourg} et \bsc{M.~Lesage} sont les fondateurs de cette
structure, le premier s'occupe de la partie technique en tant que chef
de projet et le second de la partie gestion de l'entreprise. Ils sont
accompagnés par deux développeurs, l'un travaillant sur des solutions
Windows Mobile~\textregistered, l'autre étant plus spécialisé sur le
développement Web.

Mon premier stage consistait à développer un utilitaire pour la partie
web de comparaison de bases de données qui permettrait d'améliorer le
suivi des mises à jour d'une base obsolète à partir d'une base de
référence et aussi de fournir les requêtes permettant cette mise à jour.
Mon deuxième travail était de mettre en place un nouvel outil de gestion
de version de code source plus évolué que l'existant.

Mon deuxième stage, qui est l'objet de cette note de synthèse,
consistait à continuer le travail précédent sur le gestionnaire de
version puis à crée un serveur Web dérivée d'un logiciel libre existant
pour ensuite le déployer avec une application d'\emph{IdentIt} chez un
client, le tout en s'accordant avec le système d'information de
l'acquéreur.

Cette note de synthèse suit un plan spécifique, mais les événements ne
se sont pas passés forcement dans l'ordre auquel ils sont présentées. En
effet, à cause des difficultés rencontrées et des problématiques
nouvelles données en cours de stage, je suis souvent passé d'une
réalisation à une autre.
% chapter Introduction (end)
