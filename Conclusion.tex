\chapter{Conclusion} % (fold)
\label{cha:Conclusion}

\lettrine{L}{a} première chose qui me vient à l'esprit est la
progression que m'ont apportés ces stages. Je peux dire en toute
humilité que j'en ressors grandi.

J'ai pu m'appercevoir que l'entreprise utilisait toujours les
technologies que j'ai mis en place durant la première année, des
modifications ont été aportées mais mon code de base est toujours là, ce
qui indique une certaine qualité dans le travail que j'ai pu accomplir.

J'ai continué de former les développeurs sur Git pour qu'ils aillent
toujours plus loin dans l'exploitation de celui-ci. En effet, j'avais
constaté dès la première semaine du deuxième stage que quelques
principes fondamentaux d'utilisation avait été oubliés. J'ai pu aussi
leur présenter de nouvelles fonctionnalitées qui sont apparut pendant le
laps de temps entre les deux stages. L'usage de base convient dans
90\%{} des cas mais certaines particularités sont toujours bonnes à
connaître.

J'ai découvert le travail de développeur en entreprise et je dois dire
que cela me plaît, cependant ce n'est pas toujours évident.  Bloquer sur
un problème toute la journée sans avancer est très pénible
intellectuellement, j'ai connu beaucoup de maux de tête et de matins
difficiles. Travailler toute la journée à concevoir des algorithmes est
parfois éprouvant\dots

J'ai été très satisfait des échanges que j'ai pu avoir avec l'équipe.
Développer est une chose mais proposer, débattre, trouver les meilleures
solutions sont ce que j'ai préféré. L'entreprise m'a beaucoup apporté,
mais je pense aussi avoir apporté des choses et c'est très gratifiant.

Cette période m'a prouvé que je ne m'étais pas trompé dans mon
orientation. Je souhaite continuer mes études vers une licence
professionnelle systèmes informatiques et logiciels option \og
Développement et Administration de sites Internet et Intranet \fg{} qui
permettra de me spécialiser dans le Web et la sécurité, qui sont des
marchés porteurs autant dans les entreprises publiques que privées,
avant de m'insérer dans le monde du travail.
% chapter Conclusion (end)
