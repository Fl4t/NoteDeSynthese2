\chapter{Création d'IITAMP} % (fold)
\label{cha:Création d'IITAMP}

\begin{it}
  La majeur partie des entreprises utilisant l'informatique
  dispose d'un réseau local. Le but de IITAMP\, \footnote{\emph{IdentIt
  Apache MySQL PHP.}} est de fournir les outils capables de faire
  fonctionner une application Web développé par \emph{IdentIt} sur un
  poste informatique d'une entreprise de façon simple et efficace, ce
  qui permet de ce passer d'un hébergeur sur internet et d'autres
  avantages.
\end{it}

\section{Les avantages d'un intranet} % (fold)%{{{
\label{sec:Les avantages d'un intranet}

\lettrine{C}{ertaines} entreprises dans un but de confidentialité
préfèrent utiliser des applications en intranet. Dans mon cas, un
serveur sous Windows~\textregistered{} Server 2008 utilise IITAMP ainsi
qu'une application \emph{IdentIt} pour qu'a partir du réseau local tout
les ordinateurs puissent utiliser l'application. De ce fait, le produit
n'est pas accessible via l'Internet.

La confidentialité n'est pas le seul atout. En effet, passer par un
réseau d'entreprise en terme de performance est avantageux car les
requêtes entre le client et le serveur ce font \emph{intra-muros} tandis
que passer par un hébergeur qui peut ce trouver à des milliers de
kilomètres affecte les délais. Utiliser une solution d'hébergement sur
le réseau des réseaux à un coût alors qu'une solution locale n'est pas
forfaitaire.

Dans le cas d'un hébergement, si la connexion à internet vient à être
interrompu, le service n'est plus disponible. Par contre avec notre
solution ça continue à fonctionner.
% section Les avantages d'un intranet (end)%}}}

\section{La base applicative} % (fold)%{{{
\label{sec:La base applicative}

Je suis tout d'abord parti d'une base XAMPP générique. C'est un logiciel
libre développer par une équipe de bénévoles. Cette ensemble de
programmes permettent de mettre en \oe{}uvre un serveur web sur un PC
Windows à des fins de développement. La suite XAMPP regroupe les
logiciels Apache\, \footnote{Serveur HTTP}, MySQL\, \footnote{Serveur de
base de données.} et PHP\, \footnote{\emph{PHP: Hypertext Preprocessor}
acronyme récursif désignant un langage de script.}. De ce fait, il
permet d'héberger sur la machine qui la compose un site internet
dynamique. La solution de base ne nous satisfaisant pas, mon travail
consista à modifier ce gratuiciel pour les besoins de déploiement des
produits \emph{IdentIt}.
% section La base applicative (end)%}}}

\section{Les objectifs} % (fold)%{{{
\label{sec:Les objectifs}

\subsection{La maintenabilité} % (fold)%{{{
\label{sub:La maintenabilité}

Pour proposer une maintenance de bug efficace est facile il a fallut
installer le module PHP \emph{Xdebug} qui permet d'afficher des erreurs
plus significatives que celles de PHP par défaut. En effet, celui-ci
permet la coloration syntaxique ainsi que les valeurs des variables qui
précède le bug. J'ai aussi externalisé les fichiers d'historique dans un
dossiers séparé ce qui permet d'y accéder facilement. Auquel cas on
puisse demander au client de nous envoyer le dossier archivé par
courrier électronique pour analyser si Xdebug ne suffisait pas pour
d'écrire l'erreur rencontré par téléphone.
% subsection La maintenabilité (end)%}}}

\subsection{L'évolutivité} % (fold)%{{{
\label{sub:L'évolutivité}

Comme les dossiers sont séparés méticuleusement, une mise à jour de
l'applicatif web ou du serveur IITAMP ce fait à partir en quelques
étapes très simple contrairement au logiciel de base qui aurait
nécessité beaucoup de manipulation, la plus pars de ses interventions
étant demander au client lorsque les applications seront misent en
production.
% subsection L'évolutivité (end)%}}}

\subsection{La pérennité des données} % (fold)
\label{sub:La pérennité des données}

La informations des bases de données ainsi que les informations du
navigateur Web sont extraite du dossier IITAMP ce qui permet de faire
une sauvegarde efficace de celles-ci. La sauvegarde ce fait à partir
d'une tache journalière que j'ai décrite dans un document pour
l'utilisateur.
% subsection La pérennité des données (end)

\subsection{L'éfficacité} % (fold)%{{{
\label{sub:L'éfficacité}

Pour optimiser le serveur IITAMP, une inspection profonde du logiciel
fut nécessaire. En effet, celui-ci est composé de base d'une multitude de
librairies et modules qui ne sont pas forcement utile au applications
développé par \emph{IdentIt}. De ce fait, un nettoyage fut entrepris. Un
gain de 20 mega-octets à été constaté.
% subsection L'éfficacité (end)%}}}

\subsection{La simplicité} % (fold)%{{{
\label{sub:La simplicité}

Pour rendre simple l'utilisation du serveur web pour le client, il à
fallut créer des scripts d'automatisations pour certaines taches pas
forcements évidentes pour les non initiés. Par exemple l'installation des
applications entant que services ainsi que le démarrage des dit services.
J'ai aussi développer une page d'index dans le cas ou le client à
plusieurs applications web de l'entreprise \emph{IdentIt}, auquel cas une
sélection de l'application est a faire.
% subsection La simplicité (end)%}}}

\subsection{La propriété} % (fold)%{{{
\label{sub:La propriété}

Pour la partie juridique dans un soucis de propriété intellectuelle,
j'ai installé un module qui permet le chiffrement du code source PHP,
cette extension s'appelle Bcompiler et m'a servi lors de l'installation
des applications après la finalisation du serveur IITAMP.
% subsection La propriété (end)%}}}
% section Les objectifs (end)%}}}

\section{Les documentations} % (fold)%{{{
\label{sec:Les documentations}

\begin{description}

  \item[Création d'un IITAMP :] Permet de recréer un serveur IITAMP à
    partir d'un serveur XAMPP qui incorpore des nouveautés ou des mises
    à jours.

  \item[Mise en \oe{}uvre d'un IITAMP :] mise en place de l'IITAMP
    nouvellement installé en entreprise ou mise à jour de celui-ci.

  \item[Construction d'une application Web :] Permet de guider la
    préparation d'une application Web d'\emph{IdentIt} pour quel soit
    parfaitement intégré à IITAMP.

  \item[Déploiement d'une application Web :] Comment mettre en place ou
    mettre à jour une application web dans un IITAMP déjà en service.

  \item[Maintenance de bases de données :] Explication de la mise en
    place d'une sauvegarde automatique des bases de données des
    applications ainsi que la restauration de ses sauvegardes.

  \item[Démarche d'activation d'une application GeMa :] Documentation
    pour \emph{IdentIt} compréhensible par les non techniciens qui
    explique comment activer une application chez un client.

  \item[Mise en place de GeMa en entreprise :] Documentation pour les
    clients qui décrit les premières étapes à suivre pour utiliser GeMa.

\end{description}
% section Les documentations (end)%}}}
% chapter Création d'IITAMP (end)
