\chapter{Mise en oeuvre d'IITAMP avec une solution IdentIt} % (fold)
\label{cha:Mise en oeuvre d'IITAMP avec une solution IdentIt}

\begin{it}
Après la creation d'IITAMP ainsi que l'élaboration de toute sa
documentation, j'ai installé celui-ci sur un serveur local chez un
client pour héberger la solution GeMa qui est développer par IdentIt.
J'ai eu aussi l'opportunité de réaliser une video d'utilisation de
l'application GeMa dans le cadre d'une présentation par les responsables
de l'entreprise cliente à leurs salariès.
\end{it}

\section{Présentation de GeMa} % (fold)%{{{
\label{sec:Présentation de GeMa}

GeMa est une solution de maintenance inovante alliant RFID, mobilité et
technologie WEB. L'application embarquée sur PDA permet la consultation
des interventions à réaliser, la saisie des rapports de celle-ci et
l'accès aux données techniques nécessaires au bon déroulement des
travaux.

'Avec GeMa, le responsable de maintenance prépare depuis son poste
informatique les ordres de travaux pour lesquels il mandate un
intervenant sur site. Les techniciens, quant à eux, reçoivent en temps
réel leurs missions sur PDA durcis qu'ils complètent grâce à une
interface ergonomique (menus déroulants, cases à cocher...). Les
rapports ainsi réalisés remontent instantanément sur le serveur' GeMa
exploite pleinement les nouvelles technologies. Une intervention à
réaliser sur un équipement : le technicien l'identifie grâce à son
étiquette RFID scannée par le PDA. Une anomalie découverte lors d'un
contrôle : le technicien la photographie grâce à l'appareil numérique
intégré. Un élément laissé sur place : le technicien indique sa position
sur site, grâce aux fonctions GPS de GeMa. En temps réel, les
responsables et/ou clients disposent d'informations fiables et
exploitables au travers d'historiques, de statistiques, de bilans,
d'alertes email ou via des échanges avec un logiciel tiers (ERP,
GMAO...).
% section Présentation de GeMa (end)%}}}

\section{Presentation du client Syngenta} % (fold)%{{{
\label{sec:Presentation du client Syngenta}

Le groupe Syngenta est un des leaders mondiaux de l’agrofourniture.
Syngenta développe une approche par culture reposant sur 2 piliers,
d'une part la création, le développement et la commercialisation de
variétés pour les productions agricoles majeures, d'autre part la
protection des plantes.

\begin{description}

  \item[production de variétés pour l'agriculture :] betterave à sucre,
    céréales à paille (blé tendre, blé dur, orge de printemps, orge
    hybride), fleurs, légumes (pour le marché du frais et de
    l’industrie), maïs, oléagineux (tournesol et colza) et les cultures
    légumières.

  \item[La protection des plantes :] Une combinaison de solutions
    phytopharmaceutiques contre les principaux ravageurs et nuisibles
    des cultures à base de produits de protection des semences,
    d’herbicides, de fongicides et d’insecticides mais aussi d’insectes
    auxiliaires.

\end{description}
% section Presentation du client Syngenta (end)%}}}

\section{Mise en place de la solution} % (fold)%{{{
\label{sec:Mise en place de la solution}

Les lignes de production des variétés de plantes ce trouve a deux
endroits differents en France, la mise en oeuvre d'IITAMP ainsi que de
GeMa s'éffectue dans l'usine de Nerac près de bordeaux. Pour éviter de
me déplacer de Dunkerque jusqu'au site, ce qui représente a peu pres 600
kilometres, un acces VPN à été ouvert pour me permettre d'installer à
distance le serveur IITAMP ainsi que l'application. Un dispositif de
sécurité impressionnant à été mis en place par le groupe Syngenta pour
communiquer entre les réseaux locaux des différents sites.

explication de la sécurité

\subsection{Harmonisation avec le Système d'information existant} % (fold)%{{{
\label{sub:Harmonisation avec le Système d'information existant}

Dans le cadre de cette mise en oeuvre, le client veut récuperer les
données des interventions qui sont stockés dans la base mysql de GeMa
dans leur systeme d'information qui possede une base de donnée SQL
server. Cette duplication de donnée permet une intégration des
informations récoltés par GeMa dans le SI de l'entreprise a des fins de
traçabilité, ce qui permet à partir d'autre logiciel du client de faire
des statistiques de production. Pour cette duplication, il à fallu que
je développe un script PHP qui tout dabord intéroge la base de donnée de
GeMa pour récuperer les données, qui ensuite traite ses données pour les
rendres compatibles à la structure de la base du client ainsi qu'a la
téchnologie SQL server utilisée et qui enfin insert ses données traités
dans la base du client.

\subsection{Les tests} % (fold)%{{{
\label{sub:Les tests}

Comme nous l'avons vu, GeMa utilise la mobilité pour le rapport des
interventions. Pour tester si l'application GeMa fonctionne correctement
ainsi que le script de duplication de donnée, quelques tests furent
necessaire. Étant donnée que le site de production n'est pas proche,
l'utilisation d'un émulateur de PDA au travers du reseau VPN fu
necessaire.

Quelques problèmes sont apparu pendant la phase de debogage :
\begin{itemize}

  \item Impossibilité d'utiliser le driver standard de communication
    entre mysql et SQLServer : XAMPP par defaut ne permet pas d'utiliser
    une base SQL Server, il à fallut que j'installe et configure le
    driver officiel de l'entreprise microsoft pour permettre l'accès à
    la base de donnée.

  \item Synthaxe invalide pour SQL serveur : une habitude est ancrée
    dans l'utilisation du logiciel libre mysql qui est de forcer
    l'encodage des caractères en utf-8 afin d'éviter de futur problèmes.
    Cependant SQL server ne connait pas cette syntaxe. GeMa étant pourvu
    d'un système d'alerte par mail, M.~Dubourg recevait un courriel à 5
    minutes d'intervales lorsque l'émulateur PDA était lancé,
    (represente la synchronisation automatique de GeMa PDA sur son
    socle) sont client mail à bien entendu déplacer l'éxpediteur entant
    que spam au bout de plusieurs reprises. Ne fermant pas l'émulateur
    la nuit pour éviter d'avoir à le relancer, une quantité
    impressionnante de mail m'a été annoncé quelques jours plus tard
    lorsque le co-gérant à inspecté sa boite à pourriels.

  \item Empecher les valeurs null pour les statistiques : l'ancien
    système de relevés ce faisant à l'aide d'un papier/crayon puis par
    l'ajout de ses annotations dans un fichier excel, lorsqu'un champ
    n'avait aucune valeur, un zéro lui était attribué. En inspectant les
    données de la base SQL Server, je me suis rendu compte que mon
    script quant à lui n'insérai rien lorsqu'il n'y avait pas de valeur.
    J'en ai déduit qu'aucune régle de gestion n'avait été crée afin
    d'empecher la non-saisi. J'ai donc du inscrire cette regle via le
    code (c'est la base du client, je n'ai pas a intervenir dessus même
    si elle est mal conçu) alors que la bonne methode est habituellement
    de le faire via une option lors de la création de la base.
\end{itemize}
% subsection Les tests (end)%}}}
% subsection Harmonisation avec le Système d'information existant (end)%}}}
% section Mise en place de la solution (end)%}}}
% chapter Mise en oeuvre d'IITAMP avec une solution IdentIt (end)
