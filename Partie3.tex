\chapter{Création d'IITAMP} % (fold)
\label{cha:Création d'IITAMP}

\begin{it}

  La majeure partie des entreprises utilisant l'informatique disposent
  d'un réseau local. Le but de IITAMP\, \footnote{\emph{IdentIt Apache
  MySQL PHP.}} est de fournir les outils capables de faire fonctionner
  une application Web développée par \emph{IdentIt} sur un poste
  informatique d'une entreprise de façon simple et efficace, ce qui
  permet de se passer d'un hébergeur sur internet et de bénéficier
  d'autres avantages.

\end{it}

\section{Les avantages d'un intranet} % (fold)%{{{
\label{sec:Les avantages d'un intranet}

\lettrine{C}{ertaines} entreprises, dans un but de confidentialité
préfèrent utiliser des applications en intranet. Dans le cadre de mon
stage, un serveur sous Windows Server 2008 utilise IITAMP ainsi qu'une
application d'\emph{IdentIt} pour qu'à partir du réseau local tous les
ordinateurs puissent utiliser l'application. De ce fait, le produit
n'est pas accessible via Internet.

La confidentialité n'est pas le seul atout. En effet, passer par un
réseau d'entreprises en terme de performance est avantageux car les
requêtes entre le client et le serveur se font \emph{intra muros} ;
alors que passer par un hébergeur qui peut se trouver à des milliers de
kilomètres, affecte les délais. Utiliser une solution d'hébergement sur
le réseau des réseaux a un coût alors qu'une solution locale n'est pas
forfaitaire.

Dans le cas d'un hébergement, si la connexion à internet vient à être
interrompue, le service n'est plus disponible. Par contre, avec notre
solution cela continue de fonctionner.
% section Les avantages d'un intranet (end)%}}}

\section{La base applicative} % (fold)%{{{
\label{sec:La base applicative}

Je suis tout d'abord parti d'une base XAMPP générique. Il s'agit d'un
logiciel libre développé par une équipe de bénévoles. Cet ensemble de
programmes permet de mettre en \oe{}uvre un serveur web sur un PC
Windows à des fins de développement. La suite XAMPP regroupe les
logiciels Apache\, \footnote{Serveur HTTP}, MySQL\, \footnote{Serveur de
base de données.} et PHP\, \footnote{\emph{PHP: Hypertext Preprocessor}
acronyme récursif désignant un langage de script.}. De ce fait, il
permet d'héberger sur la machine qui la compose un site internet
dynamique. La solution de base ne nous satisfaisant pas, mon travail
consista à modifier ce gratuiciel pour les besoins de déploiement des
produits \emph{IdentIt}.
% section La base applicative (end)%}}}

\section{Les objectifs} % (fold)%{{{
\label{sec:Les objectifs}

\subsection{La maintenabilité} % (fold)%{{{
\label{sub:La maintenabilité}

Pour proposer une maintenance de bug efficace et facile, il a fallu
installer le module PHP \emph{Xdebug} qui permet d'afficher des erreurs
plus significatives que celles de PHP par défaut. En effet, celui-ci
permet la coloration syntaxique ainsi que l'affichage des valeurs des
variables qui précèdent le bug. J'ai aussi externalisé les fichiers
d'historique dans un dossier séparé, ce qui permet un accès plus facile
dans le cas où il faudrait demander au client de nous envoyer le dossier
archivé par courrier électronique pour l'analyser si Xdebug ne suffisait
pas pour décrire l'erreur rencontré par téléphone.
% subsection La maintenabilité (end)%}}}

\subsection{L'évolutivité} % (fold)%{{{
\label{sub:L'évolutivité}

Comme les dossiers sont séparés méticuleusement, une mise à jour de
l'applicatif web ou du serveur IITAMP se fait à partir de quelques
étapes très simples contrairement au logiciel de base qui aurait
nécessité beaucoup de manipulations, la pluspart de ces interventions
étant demandées au client lorsque les applications seront mises en
production.
% subsection L'évolutivité (end)%}}}

\subsection{La pérennité des données} % (fold)%{{{
\label{sub:La pérennité des données}

Les informations des bases de données ainsi que celles du navigateur Web
sont extraites du dossier IITAMP ce qui permet de faire une sauvegarde
efficace de celles-ci. La sauvegarde se fait à partir d'une tache
journalière que j'ai décrite dans un document pour l'utilisateur.
% subsection La pérennité des données (end)%}}}

\subsection{L'efficacité} % (fold)%{{{
\label{sub:L'efficacité}

Pour optimiser le serveur IITAMP, une inspection profonde du logiciel
fut nécessaire. En effet, celui-ci est composé de base d'une multitude
de librairies et modules qui ne sont pas forcément utiles aux
applications développées par \emph{IdentIt}. De ce fait, un nettoyage
fut entrepris. Un gain de 20 méga-octets a été constaté.
% subsection L'efficacité (end)%}}}

\subsection{La simplicité} % (fold)%{{{
\label{sub:La simplicité}

Pour rendre simple l'utilisation du serveur web pour le client, il a
fallu créer des scripts d'automatisation pour certaines taches pas
forcément évidentes pour les non-initiés. Par exemple l'installation des
applications en tant que services ; ainsi que le démarrage des dit
services.  J'ai également développé une page d'index au cas où le client
a plusieurs applications web de l'entreprise \emph{IdentIt}, auquel cas,
une sélection de l'application est à faire.
% subsection La simplicité (end)%}}}

\subsection{La propriété} % (fold)%{{{
\label{sub:La propriété}

Pour la partie juridique, dans un souci de propriété intellectuelle,
j'ai installé un module qui permet le chiffrement du code source PHP,
cette extension s'appelle Bcompiler et m'a servi lors de l'installation
des applications après la finalisation du serveur IITAMP.
% subsection La propriété (end)%}}}
% section Les objectifs (end)%}}}

\section{Les documentations} % (fold)%{{{
\label{sec:Les documentations}

\begin{description}

  \item[Création d'un IITAMP :] Permet de recréer un serveur IITAMP à
    partir d'un serveur XAMPP qui incorpore des nouveautés ou des mises
    à jour.

  \item[Mise en \oe{}uvre d'un IITAMP :] mise en place de l'IITAMP
    nouvellement installé en entreprise ou mise à jour de celui-ci.

  \item[Préparation d'une application :] Illustre les étapes de
    préparation d'une application Web d'\emph{IdentIt} pour quelle soit
    parfaitement intégrée à IITAMP.

  \item[Déploiement d'une application :] Comment mettre en place ou
    mettre à jour une application web dans un IITAMP déjà en service.

  \item[Maintenance de bases de données :] Explication de la mise en
    place d'une sauvegarde automatique des bases de données des
    applications ainsi que la restauration de ces sauvegardes.

  \item[Démarche d'activation de GeMa :] Documentation pour
    \emph{IdentIt} compréhensible par les non techniciens qui explique
    comment activer une application chez un client.

  \item[Mise en place de GeMa en entreprise :] Documentation pour les
    clients qui décrit les premières étapes à suivre pour utiliser GeMa.

\end{description}
% section Les documentations (end)%}}}
% chapter Création d'IITAMP (end)
